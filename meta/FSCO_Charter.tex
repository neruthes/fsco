\documentclass[11pt,a4paper]{article}

\usepackage[a4paper,textwidth=37em,tmargin=24mm,bmargin=30mm]{geometry}
\usepackage{calc}
\usepackage{amsmath,xltxtra,fontspec,xunicode,tocloft,tabu,datetime2,multicol}
\setlength{\columnsep}{3em}

\usepackage{titlesec}

\usepackage{lipsum,paralist,enumitem}
\setdefaultleftmargin{2em}{2em}{1em}{1em}{1em}{1em}

\usepackage{datetime2}
\usepackage[hidelinks]{hyperref}
\hypersetup{
    colorlinks=false,
    pdfpagemode=FullScreen
}


\usepackage[PunctStyle=plain,RubberPunctSkip=false,CJKglue=\hskip 0pt,CJKecglue=\hskip 0.22em plus 200pt]{xeCJK}
% \newcommand{\myvphantom}[0]{\vphantom{QWERTYUIOPASDFGHJKLZXCVBNMqwertyuiopasdfghjklzxcvbnm1234567890ςρθδφγηξλζχψβμ汉字ひらがなカタカナ\"A}}
\newcommand{\myvphantom}[0]{\vphantom{QWERTYUIOPASDFGHJKLZXCVBNMqwertyuiopasdfghjklzxcvbnm1234567890ςρθδφγηξλζχψβμ\"A}}
\xeCJKsetup{CJKglue=\myvphantom\hskip 0pt,CJKecglue=\myvphantom\hskip 0.22em plus 200pt}
\usepackage{xeCJKfntef}
\XeTeXlinebreaklocale "zh"
\XeTeXlinebreakskip = 0pt

% =========================================
\usepackage{fancyhdr}
\usepackage{graphicx,eso-pic}

\setmainfont[Numbers=Lining]{Brygada 1918}
\setromanfont[Numbers=Lining]{Brygada 1918}
\setsansfont[Numbers=Lining]{Inter}
\setmonofont{JetBrains Mono NL}
\setCJKmainfont{Noto Serif CJK SC}
\setCJKromanfont{Noto Serif CJK SC}
\setCJKsansfont{Noto Sans CJK SC}
\setCJKmonofont{Noto Sans CJK SC}
% =========================================
\frenchspacing

\setlength{\parindent}{0pt}
\setlength{\parskip}{7pt}


\hypersetup{pdftitle={FSCO Charter}}



\begin{document}
\begin{minipage}{\linewidth}
	\parskip=12pt
	\center
	\large
	\textsc{Fictional States Cooperation Organization}

	\textsc{Organisation de Coopération des États Fictifs}

	\vskip 20pt

	\huge
	CHARTER

	\vskip 20pt
	\small
	Working Draft, 2023-01-06
\end{minipage}
\vskip 40pt



\section{Preface}
\begin{enumerate}
	\item In order to foster international cooperation and friendship, we the fictional states hereby create this organization,
	      Fictional States Cooperation Organization (French: Organisation de Coopération des États Fictifs).
	\item The working language of the Organization include English and Chinese.
\end{enumerate}


\section{Divisions}
\begin{enumerate}
	\item The Organization consists of two divisions: the General Assembly, and the Secretariat.
	\item The General Assembly is the major dialogue body of the Organization.
	      Member states hold discussions and introduce resolutions on the issues they concern.
	\item The Secretariat is chaired by the Secretary-General, who is elected by the General Assembly.
	      The Secretariat is responsible for the operational routines of the Organization.
\end{enumerate}

\subsection{General Assembly}
\begin{enumerate}
	\item The members of the General Assembly are the Ambassadors from all member states.
	      Each Ambassador participates the General Assembly on behalf of the country it represents.
\end{enumerate}

\subsection{Secretariat}
\begin{enumerate}
	\item The principal leader of the Secretariat is the Secretary-General.
	\item The Secretary-General has a term of 5 years. By the end of a term, the General Assembly may elect another Secretary-General.
	\item General Assembly may re-elect a new Secretary-General at any time.
	\item The election of Secretary-General requires at least 2 candidates. Each member state can propose 1 candidate.
	\item The Secretary-General may hire officials for individual aspects of work. Officials shall come from member states.
\end{enumerate}



\section{Finance}
\begin{enumerate}
	\item The fiscal budgets of the Organization come from member states by per-membership average.
	\item The Secretary-General is responsible for managing the expenditures of the Organization and
	      producing account books for auditions by member states.
\end{enumerate}



\end{document}
