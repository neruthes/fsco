\documentclass[a4paper,11pt]{article}
\usepackage[a4paper,textwidth=38em,vmargin=31mm]{geometry}
\usepackage{calc}
\usepackage{amsmath,xltxtra,fontspec,xunicode,tocloft,tabu,datetime2,multicol}

\usepackage{lipsum,paralist,enumitem}
\setdefaultleftmargin{2em}{2em}{1em}{1em}{1em}{1em}

\usepackage{datetime2}
\usepackage[hidelinks]{hyperref}
\hypersetup{
    colorlinks=false,
    pdfpagemode=FullScreen
}



% =========================================
\usepackage{fancyhdr}
\usepackage{graphicx,eso-pic}









\usepackage{xeCJK,xeCJKfntef}
\newcommand{\myvphantom}[0]{\vphantom{QWERTYUIOPASDFGHJKLZXCVBNMqwertyuiopasdfghjklzxcvbnm1234567890ςρθδφγηξλζχψβμ\"A}}
\xeCJKsetup{PunctStyle=plain,RubberPunctSkip=false,CJKglue=\myvphantom\hskip 0pt,CJKecglue=\myvphantom\hskip 0.22em plus 200pt}
\XeTeXlinebreaklocale "zh"
\XeTeXlinebreakskip = 0pt

\setmainfont{Brygada 1918}
\setromanfont{Brygada 1918}
\setsansfont{Nimbus Sans}
\setmonofont{CMU Typewriter Text}
\setCJKmainfont{Noto Serif CJK SC}
\setCJKromanfont{Noto Serif CJK SC}
\setCJKsansfont{Noto Sans CJK SC}
\setCJKmonofont{Noto Sans CJK SC}
% =========================================

\setlength{\parindent}{0pt}
\setlength{\parskip}{7pt}
\frenchspacing
\linespread{1.2}





\newcommand{\commonheader}[5]{
    % argv: sn, org, date, title, comments
    \hypersetup{pdftitle={#1 - #4}}
    \begin{minipage}{\linewidth}
        \normalsize
        \sffamily
        FSCO/#1\hfill#2\par\vskip 5pt
        \hrule
        \vskip 30pt
        \begin{center}
            \rmfamily
            \textsc{Fictional States Cooperation Organization}\par
            \textsc{Organisation de Coopération des États Fictifs}\par\vskip 30pt
            \parbox{\linewidth-6em}{\center\huge\sffamily\fontspec{Nimbus Roman}#4}
            \sffamily
            \par\vskip 40pt
            \normalsize#3\par\vskip 15pt
            \small\sffamily#5
        \end{center}
    \end{minipage}
    \vskip 25pt
    \hrule
    \vskip 35pt
}



\begin{document}
\commonheader{Sec/SG/2023/Note.0002}{Secretariat / Secretary-General}{2023-02-22}{Document Numbering Specification}{Publisher: Oz von Waltzdort}

\section{Preface}
\begin{enumerate}
	\item The documents produced in the Organization shall be numbered for fluent indexing and archiving.
	\item The Secretariat hereby establishes this Specification for the numbering of documents.
	\item All staff at the Secretariat and its subdivisions shall follow the guidelines specified herein.
	\item In text representation, a document ID may be prefixed by ``FSCO/''
	      to indicate that it belongs to Fictional States Cooperation Organization,
	      as a mean of avoiding ambiguity with other organizations.
\end{enumerate}

\section{General Assembly}
\begin{enumerate}
	\item A Resolution made at the General Assembly shall be numbered in the following format:
	      \texttt{GA/[YEAR]/Res.[SEQ]}, where YEAR is a 4-digit number and SEQ is an autoincremental serial number
	      (left-pad with `0' to keep it 4-digit long).
          This SEQ is scoped and is reset for each year.\\
	      Example: ``\texttt{GA/2023/Res.0001}''.
	\item When a Resolution is in draft phase, we shall add a suffix in the following format: \texttt{/D.[DRAFTSEQ]},
	      where DRAFTSEQ is an autoincremental serial number (no padding).
          This DRAFTSEQ is scoped and is reset for each Resolution.\\
	      Example: ``\texttt{GA/2023/Res.0001/D.1}''.
	\item When a meeting is held at the General Assembly, its minutes shall be numbered in the following format:
	      \texttt{GA/[YEAR]/Min.[SEQ]}, where YEAR is a 4-digit number and SEQ is an autoincremental serial number
	      (left-pad with `0' to keep it 4-digit long).
          This SEQ is scoped and is reset for each year.
	      This SEQ has no relation with the SEQ for Resolutions.\\
	      Example: ``\texttt{GA/2023/Min.0001}''.
\end{enumerate}

\end{document}
